%%%%%%%%%%%%
%PAGE SETUP%
%%%%%%%%%%%%
\documentclass{beamer}


%%%%%%%%%
%IMPORTS%
%%%%%%%%%
\usepackage{graphicx}
\graphicspath{{./Images/}}

%\usepackage{mwe} % for placeholder images

%%%%%%
%MISC%
%%%%%%
\title{\LaTeX template presentation}
\author{Author}
\date{2...-..-..}

\newcommand{\myLink}[2]{\href{#1}{\color{blue}\underline{\smash{\texttt{#2}}}}}
\newcommand{\myURL}[1]{\myLink{https://#1}{#1}}

\newcommand{\obfuscator}{\mathcal{O}}
\renewcommand{\O}{\mathcal{O}}
\newcommand{\Nat}{\mathcal{N}}
\newcommand{\Support}{\text{Supp}}
\newcommand{\poly}{\text{poly}}
\newcommand{\bits}{\{0,1\}}
\newcommand{\comment}{}
\newcommand{\inputsize}[1]{\text{input}(#1)}
\usetheme{Berlin}
% For section 3.1
\newcommand{\Cab}{C_{\alpha,\beta}}
\newcommand{\Dab}{D_{\alpha,\beta}}
\newcommand{\Fab}{F_{\alpha,\beta}}
\newcommand{\Gab}{G_{\alpha,\beta}}

\newcommand{\encode}{\text{Enc}_K}
\newcommand{\decode}{\text{Dec}_K}
\newcommand{\homoencode}{\text{Hom}_K}
\newcommand{\negligible}{\text{neg}}

%%%%%%%
%BEGIN%
%%%%%%%

\begin{document}

\maketitle

%\begin{frame}{Table of Contents}
%    \tableofcontents
%\end{frame}

%%%%%%%%%%%%%%%%%%%%%%%
\section{Background}
\begin{frame}{Introduction}
    \begin{itemize}
        \item{Obfuscation is the creation of a ``virtual black box''}
        \item<2->{Obfuscation might be easy}
        \begin{itemize}
            \item<3->Many languages are already hard to reverse engineer
            \item<4-> Halting Problem is provably impossible
            \item<5-> Rice’s Theorem
        \end{itemize}
        \item<6-> True obfuscation is impossible
    \end{itemize}

    \centering
    \includegraphics[width=\textwidth]{example-image}
\end{frame}

\begin{frame}{Applications of Obfuscation}
    %NOTE: this is just to demonstrate what can be done with beamer; it's
    %actually a terrible way of doing things because using "only" with different
    %sized things (and without anything at all on <4>) makes the bullets move
    %between slides
    \begin{itemize}
    \item<1-> Asymmetric encryption using symmetric algorithms
    \item<4-> Weakening algorithms / Anti-piracy
    \item<5-> Software watermarking
    \end{itemize}

    \only<1>{\includegraphics[width=\textwidth]{example-image-a}}
    \only<2>{\includegraphics[width=\textwidth]{example-image-b}}
    \only<3>{\includegraphics[width=\textwidth]{example-image-c}}
    \only<5>{\includegraphics[width=\textwidth]{example-image-golden}}
\end{frame}

\section{Section 2}
\begin{frame}{asdf}
    \begin{itemize}
        \item{Obfuscation is the creation of a ``virtual black box''}
        \item<2->{Obfuscation might be easy}
        \begin{itemize}
            \item<3->Many languages are already hard to reverse engineer
            \item<4-> Halting Problem is provably impossible
            \item<5-> Rice’s Theorem
        \end{itemize}
        \item<6-> True obfuscation is impossible
    \end{itemize}
\end{frame}

\begin{frame}{Prior Work}
\begin{itemize}
    \item Heuristic algorithms for obfuscations and watermarking are common
    \item Theoretical study of hardware based software protection by Goldreich and Ostrovsky
    \item Hada considered the theoretical consequences of obfuscators
    \item Canetti demonstrated by example that function definition can be more useful than just an oracle
\end{itemize}
\end{frame}

\end{document}
